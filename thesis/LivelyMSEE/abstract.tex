Traditionally, optical circuit design is tested and validated using software which implement numerical modeling techniques such as Beam Propagation, Finite Element Analysis and the Finite-Difference Time-Domain (FDTD) method.

FDTD simulations require significant computational power. Existing installations may distribute the computational requirements across large clusters of high-powered servers. This approach entails significant expense in terms of hardware, staffing and software support which may be prohibitive for some research facilities and private-sector engineering firms.

The application of modern programmable GPUs to problems in scientific visualization and computation has facilitated faster development cycles for a variety of industry segments including large dataset visualization\cite{raycasting}, aerospace\cite{Strzodka2013381} and optical circuit design. GPU-based supercomputers such as National Labs' Summit\cite{nvidiaNationalLabs}, co-designed by NVIDIA and IBM, provide dramatically increased compute capability while using less power than CPU-based solutions. 

The FDTD algorithm maps well to the massively-multithreaded data-parallel nature of GPUs. This thesis explores a GPU-based FDTD implementation and details performance gains, limitations of the GPU approach, optimization techniques and potential future enhancements. 