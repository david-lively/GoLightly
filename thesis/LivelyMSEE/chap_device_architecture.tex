\chapter{Device Architecture} \label{ch:device architecture}

CPUs and GPUs each offer advantages for different computational tasks. Multi-core CPUs offer independent cores which are effectively discrete processors. GPUs, however, offer large-scale parallelization, but require strong data and code coherence in order to achieve acceptable performance.


\section{CPU}

Users typically run many different applications in parallel: a web browser, music player, word processor and email client are a common combination.

In a modern multi-core CPU, each core provides a dedicated ALU and register set. This allows each core to operate as an independent device. This architecture is advantageous when the device is required to perform disparate operations.

Each ALU performs the same operation at the same time, indicating that the additional ALUs are redundant. Thus, the flexible, general-purpose nature of a CPU becomes a limitation when within the context of FDTD.

\section{GPU}

Unlike CPUs, GPUs are capable of running thousands of threads in parallel using a SIMD (single-instruction, multiple-data) model. This architecture is allows rapid processing of large datasets wherein each datum exhibits little or no interdependence. 

This independence is crucial to performant GPU-based applications, as detailed in \autoref{sec:simd}.

\subsection{SIMD}\label{sec:simd}

In a SIMD\cite{Massingill:2007:SAP:1772070.1772078} architecture, a core may consist of a single ALU, with multiple register banks. Separate "threads" load different data into dedicated register banks. The ALU executes identitical operations on all registers simultaneously. In this way, each register bank is analagous to a CPU "thread." However, since the same instruction must be executed on all register sets at every step, this design is less flexible than a system where each core is truly independent. 

That said, the approach provides some benefits, including:

\begin{itemize}
	\item Fewer components are required per core since fewer ALUs are required. Leads to reduced die space requirements.
	\item Better code caching. A single ALU, and corresponding cache, are used for many threads, eliminating the need to load or monitor code cache behavior per thread.
\end{itemize}





