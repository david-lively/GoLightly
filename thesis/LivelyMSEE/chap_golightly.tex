\chapter{GoLightly} \label{ch:golightly}

GoLightly is the GPU-based FDTD simulator application that is the focus and product of this thesis. Written using a combination of C++, CUDA and OpenGL, it provides a lightweight yet complete FDTD solution.

\section{Goals}

GoLightly is intended to address some issues that seem common to CPU-based solutions. In particular, it is designed to be fast, friendly and portable.


\begin{itemize}
	\item Fast. An iterative design process requires rapid feedback from the simulator. Long simulation times necessitated by existing solutions inhibit this process.
	\item Friendly. Definition of models and other simulation parameters should not require expertise in software development or quasi-proprietary scripting languages. 
	\item Portable. Ideally, the simulator should run on a high-end consumer grade laptop and support the most common desktop operating systems (Microsoft Windows and Apple OS X).  
\end{itemize}

To meet those goals, GoLightly takes advantage of the oft-underutilized programmable GPU available in common desktop and laptop computers, resulting in a dramatic speedup. Rather than relying on a proprietary model definition language or obscure, limited scripting system, we use industry-standard image and geometry file formats so that models may be defined using robust, familiar, readily-available tools. By building the software specifically for Microsoft Windows, we ensure that it is compatible with the most common desktop operating system. 

\section{Architecture}

GoLightly comprises three primary application blocks:

\begin{itemize}
	\item Model Processor
	\item Simulator
	\item Visualizer
\end{itemize}


FLOWCHART HERE?

\subsection{Model Processor}

The model processor is responsible for initialization of the simulator. At launch, the 

\subsection{Simulator}
\subsection{Visualizer}


\section{Modeling approach}
\section{Implementation}
\section{Testing methodology}
\subsection{Test Model}
\subsection{Analytical Result}
\subsection{Numerical Result}
\subsection{Comparison}
\section{Additional Examples}
\subsection{Coupler}
\subsection{Splitter}

