\chapter{Conclusions} \label{ch:conclusions}

Using the GoLightly simulator, we have shown a speedup potential of up to 12 times that achieved by CPU-based solutions.

\section{Usability}

Although the software, in its current state, may not be suitable for production, it is clear that a GPU-based approach promises substantially improved performance. This results in increased productivity and, ultimately, better products. Eliminating the script-only modeling approach enforced by other packages makes utilization of this application significantly easier, and therefore accessible to a wider audience.

\section{Future Work}

The vastly-improved performance afforded by this system presents some interesting opportunities:

\subsection{GoLightly Improvements}

Care must be taken during the design process to avoid nonsensical monitor configurations, such as disjoint or outlying pixels, or inconsistent thickness due to shape aliasing. A model validation step could correct model errors. Sub-pixel averaging would increase effective domain resolution near dielectric boundaries and reduce structure aliasing. 

\clearpage

\subsection{Genetic Algorithms}
Genetic algorithms\cite{Mitchell:1998:IGA:522098}\cite{Goldberg:1989:GAS:534133}  are intended to let software take over a part of the design process. By defining a problem in terms of a number of inputs and creating a fitness function, an application can potentially test many different designs and use a feedback loop to suggest new permutations. This approach has been shown to be successful in such diverse fields as antenna design\cite{globus2006automated}, turbine design\cite{MOSETTI1994105} and pharmaceutical research\cite{Chi:2009:MLG:1651932.1652161}. 

A fast FDTD implementation will facilitate application of this technique to optical circuit design. By defining a problem domain in terms of desired package size, available inputs, allowed waveguide shape and dielectric properties, and designing an appropriate fitness function, software will be able to rapidly evaluate different designs and suggest new permutations. This may also dramatically reduce time-to-market and reveal new research avenues. 


\subsection{Arbitrary Domain Shape and PML Sinks}
Given the flexible voxel-based model definition used in GoLightly, it becomes possible to create completely arbitrary domain shapes. Non-rectangular domains with PML "sinks" at any desired position within the domain would allow the designer to more tightly fit the computational domain to the circuit in question while ignoring irrelevant or uninteresting areas.

PML sinks would potentially increase performance by reducing memory requirements. If large sections of a domain could be surrounded by PML, those sections are effectively disjoint from the rest of the domain and therefore may be ignored and, in fact, removed from the simulation. 

Although the current implementation treats the domain as a regular grid of identical blocks, this is a programming convenience and is not dictated by the FDTD algorithm.

In addition to non-rectangular domains, non-rectangular cell shapes could improve simulation fidelity. For example, in a $TM_Z$ simulation, each $E_Z$ component could be treated as the center of a hexagonal grid cell. As such, it would be updated using three derivatives instead of two. While this would increase the computation requirement for each $E$ cell by roughly 50\% (in a two-dimensional domain), the improved fidelity may justify the cost in experiments where improved accuracy is valuable. Inclusion of the afore-mentioned PML sink capability could offset this cost. 

\subsection{Load Balancing}
Although we have shown that a GPU implementation of FDTD can outperform a CPU implementation, CPUs should not be ignored. It is possible to divide computational load between the CPU and GPU in a manner analagous to that used to distribute load between machines in a cluster. 

When combined with load balancing between separate machines, this technique would allow the GPU to act as an additional cluster node, providing a more ideal solution which, rather than trading a CPU for a GPU, utilizes the power of both. 

\subsection{Machine Learning}
A high-speed simulator such as GoLightly may facilitate application of machine learning algorithms to accelerate FDTD even more. ML systems excel at identifying complex relationships and patterns. In theory, one could present simulation parameters such as waveguide architecture and composition and source wavelengths as a set of inputs into a multi-layer neural network, while simulation output is used to evaluate the fidelity of the network's predicted output. 

A trained network may be able to predict a simulation's output with a degree of accuracy sufficient to inform the design iteration process.  The ability to rapidly execute simulations should make generation of the large quantities of training data required by ML networks more feasable than traditional CPU-based systems. 

\section{Final Words}

The field of general-purpose GPU computing offers potential yet to be realized in many of the areas where it shows the most promise. Leveraging this commonly available, underutilized resource will enable shorter, more robust design cycles, and facilitate exploration of more sophisticated waveguide architectures. 








