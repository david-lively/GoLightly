
\chapter{APPENDIX} \label{ch:table appendix}

%\vspace{-0.2in}

%\section{Description of GOY Shell Model Runs and Table}

%In most cases, we have forced the models in the first shell so that the inertial range %forms on the ultraviolet side.  However, in run 2 from Table \ref{table: app GOY table}, %we force in shell seven and observe an infra-red inertial range form.  Run 1 uses the %parameters of \cite{Yamada} originally used.  This run is used as a control and to %reproduce Pisarenko et. al. \cite{Pisarenko} work.  Run 2 is equivalent to Run 1 as far %as the parameters are concerned.  However, in this run, we force in shell 6.  As a %result, we see an infra-red inertial range.  We can apply the affine collapse to this %data as well. This result is of interest in the light of Carl Gibson idea that the true %cascade in turbulence is from small to large scales \cite{Gibson}. We start with small %forcing in Run 3.  Here, the forcing is small enough that the solution is quasi-periodic %and we have no inertial range.  We can observe this in the distribution of $u_n$.  In %this particular case, the points were located in a ring that was centered at the origin %(see Figure \ref{fig: circ symtrc}).   However, there is still circular symmetry about %the origin in this case.

%\begin{figure}[!htp]
%    \begin{center}
%        \includegraphics[width=4in]{ns_run10urui.eps}
%    \end{center}
%    \caption{Distribution of $u_i$. This plot displays the quasi-periodic nature of Run 3 %for shell 10 (eps file).} \label{fig: circ symtrc}
%\end{figure}

\section{Meep CTL Test Script}

Benchmarking Meep was performed using the CTL script:
