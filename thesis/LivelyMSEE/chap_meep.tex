\chapter{Meep} \label{ch:meep}

Meep is a full-featured, open-source simulator produced by the Massachusetts Institute of Technology. In addition to its core FDTD-based simulation engine, it provides a scripting interface for defining models and simulation parameters, recording results, and other tasks.  


\section{Modeling}

One limitation of Meep is it's use of an obscure, uncommon scripting language, SCHEME. Models are defined in terms of constructive solid geometry (CSG) commands, whereby the user describes their model in terms of boolean operations and regular polyhedra. 

While adequate for simple models, constructing an arbitrarily-shaped or dynamic structure in this way may be difficult. In practice, proprietary software may be used to convert more complex model definitions created in other software into a format usable by Meep.

It is worth noting that Meep provides a "material function" capability. This allows the user to dynamically determine the material properties of a point in space using their own algorithm rather than defining their model using CSG. 

\section{Performance}

Meep is a mature, highly-optimized suite of tools. While complex to configure and use, it scales well across multiple machines, relying on the MPI protocol to keep nodes within a simulation cluster in sync. 

While performant when compared to other FDTD software, Meep suffers from the same architecture-imposed limitations of all CPU-based implementations. The limited number of processing cores available on a general-purpose CPU restricts the number of data points that can be be processed within a given time frame. This problem can be solved by provisioning additional computers which would run in parallel, distributing the computational load evenly across the resulting cluster.

This sort of cluster configuration incurs its own overhead. Although a domain may be divided into chunks and distributed across cluster nodes, FDTD boundary conditions require that, at some point, parts of the divided domain must be exchanged between nodes to maintain continuity. This necessitates installation of a high-speed local network and supporting hardware. 

While trivial to implement for small clusters, this networking configuration can become prohibitively complex as the system scales. 

 




