\chapter{Meep} \label{ch:meep}

Meep\cite{OskooiRo10} is a full-featured, open-source electromagnetic simulator produced by the Massachusetts Institute of Technology. In addition to its core FDTD-based simulation engine, it provides a scripting interface for defining models and simulation parameters, recording results, and other tasks.  





\section{Modeling}

\iffalse
************************************************
While I may agree, you need to keep they hyperbole and sarcasm out.  You should re-write this section to state what Meep is, how it is used, and how you made use of some of it's features.  You can also look up how many citations it has (on their website) to showcase that it's widely used and trusted making it a valid point of comparison 
************************************************
\fi

One limitation of Meep is it's CSG\footnote{Constructive Solid Geometry. A method of describing manifolds as a series of boolean operations of convex polyhedra.}-based modeling language. Construction of arbitrarily-shaped or dynamic structures using CSG is a complex process. It is worth noting that Meep provides a "material function" capability. This allows the user to specify the material properties of any point in space using their own algorithm rather than defining their model using CSG. However, to take advantage of this capability, the user must employ additional software or custom programming. 

\section{Performance}

Meep comprises a mature and highly-optimized suite of tools. It scales well across multiple computers, relying on the MPI protocol to synchronize nodes within the cluster.

While performant when compared to other FDTD software, Meep suffers from the same architecture-imposed limitations of all CPU-centric implementations. The limited number of processing cores available on a general-purpose CPU restricts the number of data points that can be processed in parallel. This problem may be solved by provisioning additional computers which would run in parallel, distributing the computational load across the resulting cluster.

This sort of cluster configuration incurs its own overhead. Although a domain may be divided into discrete subdomains and distributed across cluster nodes, the state of cells located at the interfaces between subdomains must be synchronized between nodes in order to maintain continuity. This exchange must occur for every calculated time step and may necessitate use of a high-speed local network and supporting hardware to reduce update latency.

\section{Popularity}

Meep has been widely adopted by many institutions and is frequently cited in journals such as Nature\cite{vynck2012photon}\cite{krogstrup2013single}, Computer Physics Communication\cite{liu2012s}, Physical Review Letters\cite{levin2010casimir} and others. A web search revealed over 1200 citations of the original\cite{OskooiRo10} paper. This speaks to its reliability \& accuracy and indicates that it is a trusted tool.








