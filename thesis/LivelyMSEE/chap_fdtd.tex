
\chapter{FDTD} \label{ch:fdtd}

\iffalse

MPC: Here I think you need a short paragraph to tie to the work.  This work models optical fields.  FDTD is one accepted way of doing this.  Is is attractive because ...  Then what you have already says what it is.
\fi
This work models optical fields using the FDTD algorithm. FDTD eliminates any data dependence between adjacent field components, allowing them to be updated in parallel. This maps well to the SIMD GPU architecture described in the previous section. 

FDTD expresses Maxwell's equations as a set of discretized time-domain equations\cite{Yee}. These equations describe each electric field component in terms of its orthogonal, coupled magnetic fields, and each magnetic field component as a function of its coupled electric fields.


\section{Wave equation}

The TM wave equations for  $E_z$, $H_x$, and $H_y$ are of the form:

\begin{equation} \label{eq:waveequation} 
\frac{\partial E_z}{\partial t} = K * (\frac{\partial H_x}{\partial y} + \frac{\partial H_y}{\partial x})
\end{equation}

\begin{equation}
\frac{\partial H_x}{\partial t} = K * (\frac{\partial E_z}{\partial y})
\end{equation}

\begin{equation}
\frac{\partial H_y}{\partial t} = K * (\frac{\partial E_z}{\partial x})
\end{equation}

These equations state that the temporal derivative of a field is a function of the sum of the spatial derivatives of the coupled orthogonal fields.

In order to apply these equations to a computational domain, FDTD defines discretization strategies for simulated time and space. The simulation domain is divided into cells, and each frame is updated using a fixed time step derived from parameters such as source wavelength and simulation dimensionality.

\section{Yee Cell}

Yee \cite{Yee} defines a computational unit known as a "cell." The cell describes how each field component within a domain is related to it's coupled fields. For instance, each $E_Z$ field component depends on adjacent $H_y$ and $H_x$ components. The cell format in a $TM_Z$ simulation is of the form shown in \autoref{fig:yeecell}.

\begin{figure}[H]
	\centering
	\includegraphics{yee-cell-ez.png}
	\caption{2D $TM_Z$ Yee Cell}
	\label{fig:yeecell}
\end{figure}

More formally, we may expand the $E_Z$ wave equation, arriving at:

\begin{equation} \label{eq:ezupdate}
{E_z}_{i,j}^{t+1} = C_a * {E_z}_{i,j}^{t} 
+ C_b * ({H_x}_{i,j+\frac{1}{2}}^{t+\frac{1}{2}} - {H_x}_{i,j-\frac{1}{2}}^{t+\frac{1}{2}})
+ C_b * ({H_y}_{i+\frac{1}{2},j}^{t+\frac{1}{2}} - {H_xy}_{i-\frac{1}{2},j}^{t+\frac{1}{2}})
\end{equation}

Similarly, the 2D equations for the coupled fields $H_x$ and $H_y$ may be expressed as:

\begin{equation} \label{eq:hxupdate}
{H_x}_{i,j}^{t+1} = D_a * {H_x}_{i,j}^{t} + D_b * (
{E_z}_{i,j+\frac{1}{2}}^{t+\frac{1}{2}} 
-
{E_z}_{i,j-\frac{1}{2}}^{t+\frac{1}{2}}
)  
\end{equation}

\begin{equation} \label{eq:hyupdate}
{H_y}_{i,j}^{t+1} = D_a * {H_y}_{i,j}^{t} + D_b * (
{E_z}_{i+\frac{1}{2},j}^{t+\frac{1}{2}} 
-
{E_z}_{i-\frac{1}{2},j}^{t+\frac{1}{2}}
)  
\end{equation}

\iffalse
From Nathan:

You need to expand this a fair amount more.  You need to at minimum define all of the terms in these equations.  A paragraph or two on how a full domain is discritized, and how a simulation steps in 1/2 detla T steps is important.  

This will be important in the discussion between CPU and GPU architectures as you want to list how with FDTD you calcuate an entire time step, or 'frame' over the entire phyisical domain before moving on to the next frame.  You can state here how the updates of frames are independent of other calculations done during that same frame
\fi

\begin{table}[h!]
	\centering
	\caption{$TM_Z$ FDTD Equation Terms}
	\label{tab:modelColorComponentUsage}
	\begin{tabular}{l | c | l}
		Symbol	& Definition & Description \\
		\hline				\\										 	 
		$dx$ 	& $\frac{\lambda}{10}$ 			& Spatial step as a function of max source  fundamental frequency 					\\
		$dt$ 	& $\frac{dx}{\sqrt{n}} * 0.9$		& Time step between frame updates, where $n$ is the domain dimensionality \\		$C_a$	& $\frac{1}{\epsilon_0}$ & Permittivity of free space	\\
		$C_b$	& $\frac{dt}{dx}  \frac{1}{\epsilon} $ & Permittivity at the location $i,j$\\
		$D_a$	& $\frac{1}{\mu_0}$	& Permeability of free space \\
		$D_b$	& $\frac{dt}{dx}\frac{1}{\mu}$	& Permeability at the location $i,j$\\
		$i$, $j$ 	& &	Field element location within the domain  \\
		$t$   	& &	Current time step ($t_E = t_H \frac{+}{-}\frac{1}{2}$) \\
		$E_Z$ 	& & Electric field amplitude in $Z$ \\
		$H_X$ 	& & Magnetic field amplitude in $X$ \\
		$H_Y$	& & Magnetic field amplitude in $Y$ \\
	\end{tabular}
\end{table}

Since Maxwell's equations are scale-invariant, GoLightly substitutes 1 for constants such as $C$, $\mu_0$ and $\epsilon_0$. The $\frac{0.9}{\sqrt{2}}$ scalar in $dt$ prevents aliasing, and corrects for simulation dimensionality\footnote{This value should be $\frac{0.9}{\sqrt{n}}$, where $n$ is 3 for a 3D domain, 2 for a 2D domain and 1 for a 1D domain.} 

In equations \ref{eq:ezupdate}, \ref{eq:hxupdate} and \ref{eq:hyupdate}, note that each $E_{i,j}^{t+1}$ field update depends only upon the previous $E$ value ($E_{i,j}^{t}$), and the previous adjacent $H$ values ($H_X^{t+\frac{1}{2}}$ and $H_Y^{t+\frac{1}{2}}$). This independence allows each field component to be updated without regard for any other value in the same field.   

\clearpage

\section{Leap Frog: Stepping in Space and Time}

In equations \ref{eq:ezupdate}, \ref{eq:hxupdate} and \ref{eq:hyupdate}, note the presence of a "half step" in time ($F^{t + \frac{1}{2}}$) and space $F_{i \frac{+}{-}\frac{1}{2},j \frac{-}{+}\frac{1}{2}}$.

This $t\frac{+}{-}\frac{1}{2}$ represents the temporal step size between an $E$-field update and the next $H$-field update,and visa-versa. Similarly, the spatial offset $x\frac{+}{-}\frac{1}{2}$ represents the distance between an $E$-field component and its adjacent, interdependent $H$ values.


\begin{figure}[H]
	\centering
	\includegraphics[width=15cm,keepaspectratio]{YeeMesh.png}
	\caption{4x4 Yee Lattice}
	\label{fig:yeeLattice}
\end{figure}

The spatial relationship between $E$ and $H$ grids is illustrated in the Yee lattice in \autoref{fig:yeeLattice}. Note that each $E_Z$ component is in the middle of a cell, at the $(i + \frac{1}{2}, j + \frac{1}{2})$ position where ($i,j$) is  the upper-left corner of the cell. $H$ components, however, are positioned at integer coordinates.

This arrangement reflects the manner in which a wave will propagate through the domain. In the half time step between $E$ and $H$ updates, the wave moves one half-cell. If the $E$ and $H$ components were coincident, the simulation would degrade to a large collection of individual, disjoint points rather than a discretized, connected domain.


\section{Boundary Conditions}

\iffalse
MPC: I don't think I'd say enough information, I think I'd say does not contain the required information...
\fi

Recall the $H_Y$ update equation \autoref{eq:hyupdate}, and note that it depends on two adjacent $E_Z$ values on the $X$ axis. The finite grid does not contain the required information to update the $H$ values that lie on the edge of the domain. For instance, ${H_Y}_{0,\frac{1}{2}}$ requires ${E_Z}_{\frac{1}{2},\frac{1}{2}}$ and ${E_Z}_{-\frac{1}{2},\frac{1}{2}}$.

Since the position $(-\frac{1}{2},\frac{1}{2})$ does not exist, the simulator cannot update this value. The simulator must take this case into account by implementing a boundary condition, or the wave will reflect from the boundaries. 

\iffalse
Are you aware of any rule of thumb here.  What makes you think 10 is enough (I do, but you might want to justify).
\fi

We implement the Perfectly Matched Layer (PML) as described in \cite{BERENGER1994185}. A detailed explanation of PML is beyond the scope of this thesis. In practice, PML adds imaginary hyperplanes orthogonal to each field in the simulation. In the boundary regions, power couples between $E$ and $H$ as expected, but \emph{also} couples into those hyperplanes. Unlike the $E$ and $H$ interdependence, the transfer is one-way. Dielectric values in the cells in the PML describe non-physically realizable materials which force the coupled power to decay over a number of layers. In our implementation, we use 10 PML layers as experimentation showed this to provide the most satisfactory results.



\section{FDTD in SIMD}

FDTD's leap-frog update method, whereby $E$ fields and $H$ fields are successively calculated, is well-suited to a GPU implementation. $E$ field values depend on adjacent $H$ field values, and visa-versa. Since the $E$-field update equation requires knowledge only of the $H$ field state and previous $E$ field state, each field component can be calculated independently with no opportunity for a pipeline stall or race condition. 

