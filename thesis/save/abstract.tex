
Traditionally, optical circuit design is tested and validated using software which implement numerical modeling techniques such as Beam Propagation, Finite Element Analysis and FDTD.

While effective and accurate, FDTD simulations require significant computational power. Existing installations may distribute the computational requirements across large clusters of high-powered servers. This approach entails significant expense in terms of hardware, staffing and software support which may be prohibitive for some research facilities and private-sector engineering firms.

Application of modern programmable GPGPUs to problems in scientific visualization and computation has facilitated dramatically accelerated development cycles for a variety of industry segments including large dataset visualization, microprocessor design, aerospace and electromagnetic wave propagation in the context of optical circuit design.
The FDTD algorithm as envisioned by its creators maps well to the massively-multithreaded data-parallel nature of GPUs. This thesis explores a GPU FDTD implementation and details performance gains, limitations of the GPU approach, optimization techniques and potential future enhancements that may provide even greater benefits from this underutilized and often-overlooked tool. 